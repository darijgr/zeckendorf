% -------------------------------------------------------------
% NOTE ON THE DETAILED AND SHORT VERSIONS:
% -------------------------------------------------------------
% This paper comes in two versions, a detailed and a short one.
% The short version should be more than sufficient for any
% reasonable use; the detailed one was written purely to
% convince the author of its correctness.
% To switch between the two versions, find the line containing
% "\newenvironment{noncompile}{}{}" in this LaTeX file.
% Look at the two lines right beneath this line.
% To compile the detailed version, they should be as follows:
%   \includecomment{verlong}
%   \excludecomment{vershort}
% To compile the short version, they should be as follows:
%   \excludecomment{verlong}
%   \includecomment{vershort}
% As a rule, the line
%   \excludecomment{noncompile}
% should stay as it is.
% -------------------------------------------------------------
% NOTES ON SOME HACKS USED IN THIS FILE:
% -------------------------------------------------------------
% One of my pet peeves with amsthm is its use of italics in the theorem and
% proposition environments; this makes math and text indistinguishable in said
% enviroments. To avoid this, I redefine the enviroments to use the standard
% font and to use a hanging indent, along with a bold vertical bar to its
% left, to distinguish these environments from surrounding text. (Along with
% the advantage of distinguishing math from text, this also allows nesting
% several such environments inside each other, like a definition inside a
% remark. I'm not sure how good of an idea this is, though. There are also
% downsides related to the hanging indentation, such as footnotes out of it
% being painful to do right.) This is done starting from the line
%   \theoremstyle{definition}
% and until the line
%   {\end{leftbar}\end{exmp}}

\documentclass[numbers=enddot,12pt,final,onecolumn,notitlepage]{scrartcl}%
\usepackage[headsepline,footsepline,manualmark]{scrlayer-scrpage}
\usepackage[all,cmtip]{xy}
\usepackage{amssymb}
\usepackage{amsmath}
\usepackage{amsthm}
\usepackage{framed}
\usepackage{comment}
\usepackage{color}
\usepackage[sc]{mathpazo}
\usepackage[T1]{fontenc}
\usepackage{needspace}
\usepackage{tabls}
\usepackage[breaklinks=true]{hyperref}
%TCIDATA{OutputFilter=latex2.dll}
%TCIDATA{Version=5.50.0.2960}
%TCIDATA{LastRevised=Thursday, February 22, 2018 21:12:23}
%TCIDATA{SuppressPackageManagement}
%TCIDATA{<META NAME="GraphicsSave" CONTENT="32">}
%TCIDATA{<META NAME="SaveForMode" CONTENT="1">}
%TCIDATA{BibliographyScheme=Manual}
%BeginMSIPreambleData
\providecommand{\U}[1]{\protect\rule{.1in}{.1in}}
%EndMSIPreambleData
\newcounter{exer}
\newcounter{exera}
\numberwithin{exer}{section}
\theoremstyle{definition}
\newtheorem{theo}{Theorem}[section]
\newenvironment{theorem}[1][]
{\begin{theo}[#1]\begin{leftbar}}
{\end{leftbar}\end{theo}}
\newtheorem{lem}[theo]{Lemma}
\newenvironment{lemma}[1][]
{\begin{lem}[#1]\begin{leftbar}}
{\end{leftbar}\end{lem}}
\newtheorem{prop}[theo]{Proposition}
\newenvironment{proposition}[1][]
{\begin{prop}[#1]\begin{leftbar}}
{\end{leftbar}\end{prop}}
\newtheorem{defi}[theo]{Definition}
\newenvironment{definition}[1][]
{\begin{defi}[#1]\begin{leftbar}}
{\end{leftbar}\end{defi}}
\newtheorem{remk}[theo]{Remark}
\newenvironment{remark}[1][]
{\begin{remk}[#1]\begin{leftbar}}
{\end{leftbar}\end{remk}}
\newtheorem{coro}[theo]{Corollary}
\newenvironment{corollary}[1][]
{\begin{coro}[#1]\begin{leftbar}}
{\end{leftbar}\end{coro}}
\newtheorem{conv}[theo]{Convention}
\newenvironment{condition}[1][]
{\begin{conv}[#1]\begin{leftbar}}
{\end{leftbar}\end{conv}}
\newtheorem{quest}[theo]{Question}
\newenvironment{algorithm}[1][]
{\begin{quest}[#1]\begin{leftbar}}
{\end{leftbar}\end{quest}}
\newtheorem{warn}[theo]{Warning}
\newenvironment{conclusion}[1][]
{\begin{warn}[#1]\begin{leftbar}}
{\end{leftbar}\end{warn}}
\newtheorem{conj}[theo]{Conjecture}
\newenvironment{conjecture}[1][]
{\begin{conj}[#1]\begin{leftbar}}
{\end{leftbar}\end{conj}}
\newtheorem{exam}[theo]{Example}
\newenvironment{example}[1][]
{\begin{exam}[#1]\begin{leftbar}}
{\end{leftbar}\end{exam}}
\newtheorem{exmp}[exer]{Exercise}
\newenvironment{exercise}[1][]
{\begin{exmp}[#1]\begin{leftbar}}
{\end{leftbar}\end{exmp}}
\newenvironment{statement}{\begin{quote}}{\end{quote}}
\iffalse
\newenvironment{proof}[1][Proof]{\noindent\textbf{#1.} }{\ \rule{0.5em}{0.5em}}
\fi
\let\sumnonlimits\sum
\let\prodnonlimits\prod
\let\cupnonlimits\bigcup
\let\capnonlimits\bigcap
\renewcommand{\sum}{\sumnonlimits\limits}
\renewcommand{\prod}{\prodnonlimits\limits}
\renewcommand{\bigcup}{\cupnonlimits\limits}
\renewcommand{\bigcap}{\capnonlimits\limits}
\setlength\tablinesep{3pt}
\setlength\arraylinesep{3pt}
\setlength\extrarulesep{3pt}
\voffset=0cm
\hoffset=-0.7cm
\setlength\textheight{22.5cm}
\setlength\textwidth{15.5cm}
\newenvironment{verlong}{}{}
\newenvironment{vershort}{}{}
\newenvironment{noncompile}{}{}
\excludecomment{verlong}
\includecomment{vershort}
\excludecomment{noncompile}
\newcommand{\id}{\operatorname{id}}
\ihead{Zeckendorf family identities generalized}
\ohead{page \thepage}
\cfoot{}
\begin{document}

\title{Zeckendorf family identities generalized}
\author{Darij Grinberg}
\date{
%TCIMACRO{\TeXButton{today}{\today}}%
%BeginExpansion
\today
%EndExpansion
, brief version}
\maketitle

\begin{abstract}
\textbf{Abstract.} In \cite{1}, Philip Matchett Wood and Doron Zeilberger have
constructed identities for the Fibonacci numbers $f_{n}$ of the form%
\begin{align*}
1f_{n}  &  =f_{n}\text{ for all }n\geq1;\\
2f_{n}  &  =f_{n-2}+f_{n+1}\text{ for all }n\geq3;\\
3f_{n}  &  =f_{n-2}+f_{n+2}\text{ for all }n\geq3;\\
4f_{n}  &  =f_{n-2}+f_{n}+f_{n+2}\text{ for all }n\geq3;\\
&  \text{etc.;}\\
kf_{n}  &  =\sum_{s\in S_{k}}f_{n+s}\text{ for all }n>\max\left\{  -s\mid s\in
S_{k}\right\}  \text{,}%
\end{align*}
where $S_{k}$ is a fixed ``lacunar'' set of integers (``lacunar'' means that
no two elements of this set are consecutive integers) depending only on $k$.
(The condition $n>\max\left\{  -s\mid s\in S_{k}\right\}  $ is only to make
sure that all addends $f_{n+s}$ are well-defined. If the Fibonacci sequence is
properly continued to the negative, this condition drops out.)\newline In this
note we prove a generalization of these identities: For any family $\left(
a_{1},a_{2},...,a_{p}\right)  $ of integers, there exists one and only one
finite lacunar set $S$ of integers such that every $n$ (high enough to make
the Fibonacci numbers in the equation below well-defined) satisfies
\[
f_{n+a_{1}}+f_{n+a_{2}}+...+f_{n+a_{p}}=\sum\limits_{s\in S}f_{n+s}.
\]
The proof uses the Fibonacci-approximating properties of the golden ratio. It
would be interesting to find a purely combinatorial proof.

\end{abstract}

%NOT proofreading completed


\hrule


\begin{statement}
This is a brief version of my note \cite{this.long}. For a long version,
which gives more details in the proofs, see \cite{this.long}.
\end{statement}

\hrule

\section{The main result}

The purpose of this note is to establish a generalization of the so-called
\textit{Zeckendorf family identities} which were discussed in \cite{1}.
Before we can state it, we need a few definitions:

\begin{definition}
A subset $S$ of $\mathbb{Z}$ is called \textit{lacunar} if it satisfies
$\left(  s+1\notin S\text{ for every }s\in S\right)  $.
\end{definition}

\begin{definition}
Let $\left(  f_{1},f_{2},f_{3},\ldots\right)  $ be the Fibonacci sequence
(defined by $f_{1}=f_{2}=1$ and the recurrence relation $\left(  f_{n}%
=f_{n-1}+f_{n-2}\text{ for all }n\in\mathbb{N}\text{ satisfying }%
n\geq3\right)  $).
\end{definition}

(Here and in the following, $\mathbb{N}$ denotes the set
$\left\{0,1,2,\ldots\right\}$.)

Our main theorem is the following:

\begin{theorem}
[generalized Zeckendorf family identities]\label{thm.1} Let $T$ be a finite
set, and $a_{t}$ be an integer for every $t\in T$.

Then, there exists one and only one finite lacunar subset $S$ of $\mathbb{Z}$
such that\footnote{Here and in the following, $\max\varnothing$ is understood
to be $0$.}
\[
\left(
\begin{array}
[c]{c}%
\sum\limits_{t\in T}f_{n+a_{t}}=\sum\limits_{s\in S}f_{n+s}\text{ for every
}n\in\mathbb{Z}\text{ which}\\
\text{satisfies }n>\max\left(  \left\{  -a_{t}\mid t\in T\right\}
\cup\left\{  -s\mid s\in S\right\}  \right)
\end{array}
\right)  .
\]

\end{theorem}

\begin{remark}
\textbf{1)} The \textit{Zeckendorf family identities}\footnote{The first seven
of these identities are%
\begin{align*}
1f_{n}  &  =f_{n}\text{ for all }n\geq1;\\
2f_{n}  &  =f_{n-2}+f_{n+1}\text{ for all }n\geq3;\\
3f_{n}  &  =f_{n-2}+f_{n+2}\text{ for all }n\geq3;\\
4f_{n}  &  =f_{n-2}+f_{n}+f_{n+2}\text{ for all }n\geq3;\\
5f_{n}  &  =f_{n-4}+f_{n-1}+f_{n+3}\text{ for all }n\geq5;\\
6f_{n}  &  =f_{n-4}+f_{n+1}+f_{n+3}\text{ for all }n\geq5;\\
7f_{n}  &  =f_{n-4}+f_{n+4}\text{ for all }n\geq5.
\end{align*}
} from \cite{1} are the result of applying Theorem~\ref{thm.1} to the case
when all $a_{t}$ are $=0$.

\textbf{2)} The condition $n>\max\left(  \left\{  -a_{t}\mid t\in T\right\}
\cup\left\{  -s\mid s\in S\right\}  \right)  $ in Theorem~\ref{thm.1} is just
a technical condition made in order to ensure that the Fibonacci numbers
$f_{n+a_{t}}$ for all $t\in T$ and $f_{n+s}$ for all $s\in S$ are
well-defined. (If we would define the Fibonacci numbers $f_{n}$ for integers
$n\leq0$ by extending the recurrence relation $f_{n}=f_{n-1}+f_{n-2}$ ``to the
left'', then we could drop this condition.)
\end{remark}

The proof we shall give for Theorem~\ref{thm.1} is not combinatorial.
It will use inequalities and the properties of the golden ratio; in a sense,
its underlying ideas come from analysis (although it will not actually use
any results from analysis).

\section{Basics on the Fibonacci sequence}

We begin with some lemmas and notations:

We denote by $\mathbb{N}$ the set $\left\{  0,1,2,...\right\}  $ (and not the
set $\left\{  1,2,3,...\right\}  $, like some other authors do). Also, we
denote by $\mathbb{N}_{2}$ the set $\left\{  2,3,4,...\right\}  =\mathbb{N}%
\setminus\left\{  0,1\right\}  $.

Also, let $\phi=\dfrac{1+\sqrt{5}}{2}$. We notice that $\phi\approx1.618...$
and that $\phi^{2}=\phi+1$.

First, some basic (and known) lemmas on the Fibonacci sequence:

\begin{lemma}
\label{lem.2} If $S$ is a finite lacunar subset of $\mathbb{N}_{2}$, then
$\sum\limits_{t\in S}f_{t}<f_{\max S+1}$.
\end{lemma}

\begin{proof}
This is rather clear either by a telescoping sum argument (write the set $S$
in the form $\left\{  s_{1},s_{2},...,s_{k}\right\}  $ with $s_{1}%
<s_{2}<...<s_{k}$, notice that
\[
\sum\limits_{t\in S}f_{t}=\sum\limits_{i=1}^{k}f_{s_{i}}=\sum\limits_{i=1}%
^{k}\left(  f_{s_{i}+1}-f_{s_{i}-1}\right)  =\sum\limits_{i=1}^{k-1}\left(
f_{s_{i}+1}-f_{s_{i+1}-1}\right)  +\underbrace{f_{s_{k}+1}}_{=f_{\max S+1}%
}-\underbrace{f_{s_{1}-1}}_{>0},
\]
and use $s_{i}+1\leq s_{i+1}-1$ since the set $S$ is lacunar) or by induction
over $\max S$ (use $f_{\max S+1}=f_{\max S}+f_{\max S-1}$ here).
\end{proof}

\begin{lemma}
[existence part of the Zeckendorf theorem]\label{lem.3} For every nonnegative
integer $n$, there exists a finite lacunar subset $T$ of $\mathbb{N}_{2}$ such
that $n=\sum\limits_{t\in T}f_{t}$.
\end{lemma}

\begin{proof}
Induction over $n$. The main idea here is to let $t_{1}$ be the maximal
$\tau\in\mathbb{N}_{2}$ satisfying $f_{\tau}\leq n$, and apply
Lemma~\ref{lem.3} to $n-t_{1}$ instead of $n$. The details are left to the
reader (and can be found in \cite{this.long}).
\end{proof}

\begin{lemma}
[uniqueness part of the Zeckendorf theorem]\label{lem.4} Let $n\in\mathbb{N}$,
and let $T$ and $T^{\prime}$ be two finite lacunar subsets of $\mathbb{N}_{2}$
such that $n=\sum\limits_{t\in T}f_{t}$ and $n=\sum\limits_{t\in T^{\prime}%
}f_{t}$. Then, $T=T^{\prime}$.
\end{lemma}

\begin{proof}
Induction over $n$. Use Lemma~\ref{lem.2} to show that $\max T<\max T^{\prime
}+1$ and $\max T^{\prime}<\max T+1$, resulting in $\max T=\max T^{\prime}$.
Hence, the sets $T$ and $T^{\prime}$ have an element in common, and we can
reduce the situation to one with a smaller $n$ by removing this common element
from both sets.
\end{proof}

Lemmata~\ref{lem.3} and~\ref{lem.4} together yield the following theorem:

\begin{theorem}
[Zeckendorf theorem]\label{thm.5} For every nonnegative integer $n$, there
exists one and only one finite lacunar subset $T$ of $\mathbb{N}_{2}$ such
that $n=\sum\limits_{t\in T}f_{t}$.

We will denote this set $T$ by $Z_{n}$. Thus, $n=\sum\limits_{t\in Z_{n}}%
f_{t}$.
\end{theorem}

\section{Inequalities for the golden ratio}

Next, we state a completely straightforward (and well-known) theorem,
which shows that the Fibonacci sequence grows roughly exponentially
(with the exponent being the golden ratio $\phi$):

\begin{theorem}
\label{thm.6} For every $n\in\mathbb{N}_{2}$, we have $\left\vert f_{n+1}-\phi
f_{n}\right\vert =\dfrac{1}{\sqrt{5}}\left(  \phi-1\right)  ^{n}$.
\end{theorem}

\begin{proof}
Binet's formula yields $f_{n}=\dfrac{\phi^{n}-\phi^{-n}}{\sqrt{5}}$ and
$f_{n+1}=\dfrac{\phi^{n+1}-\phi^{-\left(  n+1\right)  }}{\sqrt{5}}$; the rest
is computation.
\end{proof}

Let us show yet another lemma for later use:

\begin{lemma}
\label{lem.7} If $S$ is a finite lacunar subset of $\mathbb{N}_{2}$, then
$\sum\limits_{s\in S}\left(  \phi-1\right)  ^{s}\leq\phi-1$.
\end{lemma}

\begin{proof}
[Proof of Lemma~\ref{lem.7}.]Since $S$ is a lacunar subset of $\mathbb{N}_{2}%
$, the smallest element of $S$ is at least $2$, the second smallest element of
$S$ is at least $4$ (since it is larger than the smallest element by at least
$2$), the third smallest element of $S$ is at least $6$ (since it is larger
than the second smallest element by at least $2$), and so on. Since
$\mathbb{N}\rightarrow\mathbb{R}$, $s\mapsto\left(  \phi-1\right)  ^{s}$ is a
weakly decreasing function (as $0\leq\phi-1\leq1$), we thus have%
\[
\sum_{s\in S}\left(  \phi-1\right)  ^{s}\leq\sum_{s\in\left\{
2,4,6,...\right\}  }\left(  \phi-1\right)  ^{s}=\sum_{t\in\left\{
1,2,3,...\right\}  }\left(  \phi-1\right)  ^{2t}=\phi-1
\]
(by the formula for the sum of the geometric series, along with some
computations). This proves Lemma~\ref{lem.7}.
\end{proof}

\section{Proving Theorem~\ref{thm.1}}

Let us now come to the proof of Theorem~\ref{thm.1}. First, we formulate the
existence part of this theorem:

\begin{theorem}
[existence part of the generalized Zeckendorf family identities]\label{thm.8}
Let $T$ be a finite set, and $a_{t}$ be an integer for every $t\in T$.

Then, there exists a finite lacunar subset $S$ of $\mathbb{Z}$ such that
\[
\left(
\begin{array}
[c]{c}%
\sum\limits_{t\in T}f_{n+a_{t}}=\sum\limits_{s\in S}f_{n+s}\text{ for every
}n\in\mathbb{Z}\text{ which}\\
\text{satisfies }n>\max\left(  \left\{  -a_{t}\mid t\in T\right\}
\cup\left\{  -s\mid s\in S\right\}  \right)
\end{array}
\right)  .
\]

\end{theorem}

Before we start proving this, we need a new notation:

\begin{definition}
Let $K$ be a subset of $\mathbb{Z}$, and $a\in\mathbb{Z}$. Then, $K+a$ will
denote the subset $\left\{  k+a\ \mid\ k\in K\right\}  $ of $\mathbb{Z}$.
\end{definition}

Clearly, $\left(  K+a\right)  +b=K+\left(  a+b\right)  $ for any two integers
$a$ and $b$. Also, $K+0=K$. Finally, if $K$ is a lacunar subset of
$\mathbb{Z}$, and if $a\in\mathbb{Z}$, then $K+a$ is lacunar as well.

\begin{proof}
[Proof of Theorem~\ref{thm.8}.]Choose a high enough integer $N$. What exactly
``high enough'' means we will see later; at the moment, we only require
$N\in\mathbb{N}_{2}$ and $N>\max\left\{  -a_{t}\mid t\in T\right\}  $. We
might later want $N$ to be even higher, however.

Let $\nu=\sum\limits_{t\in T}f_{N+a_{t}}$. Then, Lemma~\ref{lem.3} yields
$\nu=\sum\limits_{t\in Z_{\nu}}f_{t}$ for a finite lacunar subset $Z_{\nu}$ of
$\mathbb{N}_{2}$. Let $S=\left\{  t-N\ \mid\ t\in Z_{\nu}\right\}  $. Then,
$S=Z_{\nu}+\left(  -N\right)  $ is a finite lacunar subset of $\mathbb{Z}$,
and $\nu=\sum\limits_{t\in Z_{\nu}}f_{t}$ becomes $\nu=\sum\limits_{s\in
S}f_{N+s}$. So now we know that $\sum\limits_{t\in T}f_{N+a_{t}}%
=\sum\limits_{s\in S}f_{N+s}$ (because both sides of this equation equal $\nu$).

So, we have chosen a high $N$ and found a finite lacunar subset $S$ of
$\mathbb{Z}$ which satisfies $\sum\limits_{t\in T}f_{N+a_{t}}=\sum
\limits_{s\in S}f_{N+s}$. But Theorem~\ref{thm.8} is not proven yet:
Theorem~\ref{thm.8} requires us to prove that there exists \textit{one} finite
lacunar subset $S$ of $\mathbb{Z}$ which works for \textit{every} $N$, while
at the moment we cannot be sure yet whether different $N$'s wouldn't produce
\textit{different} sets $S$. And, in fact, different $N$'s \textit{can}
produce different sets $S$, but (fortunately!) only if the $N$'s are too
small. If we take $N$ high enough, the set $S$ that we obtained turns out to
be \textit{universal}, i. e. it satisfies
\begin{align}
\sum\limits_{t\in T}f_{n+a_{t}}  &  =\sum\limits_{s\in S}f_{n+s}%
\ \ \ \ \ \ \ \ \ \ \text{for every }n\in\mathbb{Z}\text{ which}\nonumber\\
\text{satisfies }n  &  >\max\left(  \left\{  -a_{t}\mid t\in T\right\}
\cup\left\{  -s\mid s\in S\right\}  \right)  . \label{BigLemma}%
\end{align}
We are now going to prove this.

In order to prove (\ref{BigLemma}), we need two assertions:

\textit{Assertion 1:} If some $n\in\mathbb{Z}$ satisfies $n\geq N$ and
$\sum\limits_{t\in T}f_{n+a_{t}}=\sum\limits_{s\in S}f_{n+s}$, then
$\sum\limits_{t\in T}f_{\left(  n+1\right)  +a_{t}}=\sum\limits_{s\in
S}f_{\left(  n+1\right)  +s}$.

\textit{Assertion 2:} If some $n\in\mathbb{Z}$ satisfies $\sum\limits_{t\in
T}f_{n+a_{t}}=\sum\limits_{s\in S}f_{n+s}$ and $\sum\limits_{t\in T}f_{\left(
n+1\right)  +a_{t}}=\sum\limits_{s\in S}f_{\left(  n+1\right)  +s}$, then
$\sum\limits_{t\in T}f_{\left(  n-1\right)  +a_{t}}=\sum\limits_{s\in
S}f_{\left(  n-1\right)  +s}$ (if $n-1>\max\left(  \left\{  -a_{t}\mid t\in
T\right\}  \cup\left\{  -s\mid s\in S\right\}  \right)  $).

Obviously, Assertion 1 yields (by induction) that $\sum\limits_{t\in
T}f_{n+a_{t}}=\sum\limits_{s\in S}f_{n+s}$ for every $n\geq N$, and Assertion
2 then finishes off the remaining $n$'s (by backwards induction, or, to be
more precise, by an induction step from $n+1$ and $n$ to $n-1$). Thus, once
both Assertions 1 and 2 are proven, (\ref{BigLemma}) will follow and thus
Theorem~\ref{thm.8} will be proven.

Assertion 2 is almost trivial (just notice that%
\[
\sum\limits_{t\in T}f_{\left(  n-1\right)  +a_{t}}=\sum\limits_{t\in
T}\underbrace{f_{n+a_{t}-1}}_{=f_{n+a_{t}+1}-f_{n+a_{t}}}=\sum\limits_{t\in
T}f_{n+a_{t}+1}-\sum\limits_{t\in T}f_{n+a_{t}}=\sum\limits_{t\in T}f_{\left(
n+1\right)  +a_{t}}-\sum\limits_{t\in T}f_{n+a_{t}}%
\]
and%
\[
\sum\limits_{s\in S}f_{\left(  n-1\right)  +s}=\sum\limits_{s\in
S}\underbrace{f_{n+s-1}}_{=f_{n+s+1}-f_{n+s}}=\sum\limits_{s\in S}%
f_{n+s+1}-\sum\limits_{s\in S}f_{n+s}=\sum\limits_{s\in S}f_{\left(
n+1\right)  +s}-\sum\limits_{s\in S}f_{n+s}%
\]
), so it only remains to prove Assertion 1.

So let us prove Assertion 1. Here we are going to use that $N$ is high enough
(because otherwise, Assertion 1 wouldn't hold). We have $\sum\limits_{t\in
T}f_{n+a_{t}}=\sum\limits_{s\in S}f_{n+s}$ by assumption, so that
$\sum\limits_{t\in T}f_{n+a_{t}}-\sum\limits_{s\in S}f_{n+s}=0$. Thus,%
\begin{align*}
\sum\limits_{t\in T}f_{\left(  n+1\right)  +a_{t}}-\sum\limits_{s\in
S}f_{\left(  n+1\right)  +s}  &  =\sum\limits_{t\in T}f_{\left(  n+1\right)
+a_{t}}-\sum\limits_{s\in S}f_{\left(  n+1\right)  +s}-\phi\left(
\sum\limits_{t\in T}f_{n+a_{t}}-\sum\limits_{s\in S}f_{n+s}\right) \\
&  =\sum\limits_{t\in T}\left(  f_{\left(  n+1\right)  +a_{t}}-\phi
f_{n+a_{t}}\right)  -\sum\limits_{s\in S}\left(  f_{\left(  n+1\right)
+s}-\phi f_{n+s}\right) \\
&  =\sum\limits_{t\in T}\left(  f_{n+a_{t}+1}-\phi f_{n+a_{t}}\right)
-\sum\limits_{s\in S}\left(  f_{n+s+1}-\phi f_{n+s}\right)  ,
\end{align*}
so that%
\begin{align}
&  \left\vert \sum\limits_{t\in T}f_{\left(  n+1\right)  +a_{t}}%
-\sum\limits_{s\in S}f_{\left(  n+1\right)  +s}\right\vert =\left\vert
\sum\limits_{t\in T}\left(  f_{n+a_{t}+1}-\phi f_{n+a_{t}}\right)
-\sum\limits_{s\in S}\left(  f_{n+s+1}-\phi f_{n+s}\right)  \right\vert
\nonumber\\
&  \leq\sum\limits_{t\in T}\left\vert f_{n+a_{t}+1}-\phi f_{n+a_{t}%
}\right\vert +\sum\limits_{s\in S}\left\vert f_{n+s+1}-\phi f_{n+s}\right\vert
\ \ \ \ \ \ \ \ \ \ \left(  \text{by the triangle inequality}\right)
\nonumber\\
&  =\dfrac{1}{\sqrt{5}}\sum\limits_{t\in T}\left(  \phi-1\right)  ^{n+a_{t}%
}+\dfrac{1}{\sqrt{5}}\sum\limits_{s\in S}\left(  \phi-1\right)  ^{n+s}%
\ \ \ \ \ \ \ \ \ \ \left(  \text{by Theorem~\ref{thm.6}}\right) \nonumber\\
&  <\sum\limits_{t\in T}\left(  \phi-1\right)  ^{n+a_{t}}+\sum\limits_{s\in
S}\left(  \phi-1\right)  ^{n+s}\ \ \ \ \ \ \ \ \ \ \left(  \text{since }%
\dfrac{1}{\sqrt{5}}<1\right) \nonumber\\
&  \leq\sum\limits_{t\in T}\left(  \phi-1\right)  ^{N+a_{t}}+\sum\limits_{s\in
S}\left(  \phi-1\right)  ^{N+s}\ \ \ \ \ \ \ \ \ \ \left(
\begin{array}
[c]{c}%
\text{since }\left(  \phi-1\right)  ^{n+a_{t}}\leq\left(  \phi-1\right)
^{N+a_{t}}\text{ and}\\
\left(  \phi-1\right)  ^{n+s}\leq\left(  \phi-1\right)  ^{N+s}\text{,
because}\\
n\geq N\text{ and }0\leq\phi-1\leq1
\end{array}
\right) \nonumber\\
&  =\sum\limits_{t\in T}\left(  \phi-1\right)  ^{N+a_{t}}+\sum\limits_{t\in
Z_{\nu}}\left(  \phi-1\right)  ^{t}\ \ \ \ \ \ \ \ \ \ \left(  \text{since
}S=\left\{  t-N\ \mid\ t\in Z_{\nu}\right\}  \right) \nonumber\\
&  =\sum\limits_{t\in T}\left(  \phi-1\right)  ^{N+a_{t}}+\sum\limits_{s\in
Z_{\nu}}\left(  \phi-1\right)  ^{s}=\left(  \phi-1\right)  ^{N}\sum
\limits_{t\in T}\left(  \phi-1\right)  ^{a_{t}}+\sum\limits_{s\in Z_{\nu}%
}\left(  \phi-1\right)  ^{s}\nonumber\\
&  \leq\left(  \phi-1\right)  ^{N}\sum\limits_{t\in T}\left(  \phi-1\right)
^{a_{t}}+\left(  \phi-1\right)  \label{Estimate}%
\end{align}
(since $\sum\limits_{s\in Z_{\nu}}\left(  \phi-1\right)  ^{s}\leq\phi-1$ by
Lemma~\ref{lem.7}, because $Z_{\nu}$ is a lacunar subset of $\mathbb{N}_{2}$).

Now, $\sum\limits_{t\in T}\left(  \phi-1\right)  ^{a_{t}}$ is a constant,
while $\left(  \phi-1\right)  ^{N}\rightarrow0$ for $N\rightarrow\infty$.
Hence, we can make the product $\left(  \phi-1\right)  ^{N}\sum\limits_{t\in
T}\left(  \phi-1\right)  ^{a_{t}}$ arbitrarily close to $0$ if we choose $N$
high enough. Since $\phi-1<1$, we have%
\begin{equation}
\left(  \phi-1\right)  ^{N}\sum\limits_{t\in T}\left(  \phi-1\right)  ^{a_{t}%
}+\left(  \phi-1\right)  <1 \label{BoundWIN}%
\end{equation}
if $\left(  \phi-1\right)  ^{N}\sum\limits_{t\in T}\left(  \phi-1\right)
^{a_{t}}$ is sufficiently close to $0$, what we can enforce by taking a high
enough $N$. This is exactly the point where we require $N$ to be high enough.

So let us take $N$ high enough so that (\ref{BoundWIN}) holds. Combined with
(\ref{Estimate}), it then yields%
\[
\left\vert \sum\limits_{t\in T}f_{\left(  n+1\right)  +a_{t}}-\sum
\limits_{s\in S}f_{\left(  n+1\right)  +s}\right\vert <1,
\]
which leads to $\left\vert \sum\limits_{t\in T}f_{\left(  n+1\right)  +a_{t}%
}-\sum\limits_{s\in S}f_{\left(  n+1\right)  +s}\right\vert =0$ (since
$\left\vert \sum\limits_{t\in T}f_{\left(  n+1\right)  +a_{t}}-\sum
\limits_{s\in S}f_{\left(  n+1\right)  +s}\right\vert $ is a nonnegative
integer). In other words, $\sum\limits_{t\in T}f_{\left(  n+1\right)  +a_{t}%
}=\sum\limits_{s\in S}f_{\left(  n+1\right)  +s}$. This completes the proof of
Assertion 1, and, with it, the proof of Theorem~\ref{thm.8}.
\end{proof}

All that remains now is the (rather trivial) uniqueness part of
Theorem~\ref{thm.1}:

\begin{lemma}
[uniqueness part of the generalized Zeckendorf family identities]\label{lem.9}
Let $T$ be a finite set, and $a_{t}$ be an integer for every $t\in T$.

Let $S$ be a finite lacunar subset of $\mathbb{Z}$ such that%
\[
\left(
\begin{array}
[c]{c}%
\sum\limits_{t\in T}f_{n+a_{t}}=\sum\limits_{s\in S}f_{n+s}\text{ for every
}n\in\mathbb{Z}\text{ which}\\
\text{satisfies }n>\max\left(  \left\{  -a_{t}\mid t\in T\right\}
\cup\left\{  -s\mid s\in S\right\}  \right)
\end{array}
\right)  .
\]
Let $S^{\prime}$ be a finite lacunar subset of $\mathbb{Z}$ such that%
\[
\left(
\begin{array}
[c]{c}%
\sum\limits_{t\in T}f_{n+a_{t}}=\sum\limits_{s\in S^{\prime}}f_{n+s}\text{ for
every }n\in\mathbb{Z}\text{ which}\\
\text{satisfies }n>\max\left(  \left\{  -a_{t}\mid t\in T\right\}
\cup\left\{  -s\mid s\in S^{\prime}\right\}  \right)
\end{array}
\right)  .
\]
Then, $S=S^{\prime}$.
\end{lemma}

\begin{proof}
[Proof of Lemma~\ref{lem.9}.]Let%
\begin{equation}
n=\max\left(  \left\{  -a_{t}\mid t\in T\right\}  \cup\left\{  -s\mid s\in
S\right\}  \cup\left\{  -s\mid s\in S^{\prime}\right\}  \right)  +2.
\label{l7}%
\end{equation}
Then, $n$ satisfies $n>\max\left(  \left\{  -a_{t}\mid t\in T\right\}
\cup\left\{  -s\mid s\in S\right\}  \right)  $, so that
\begin{align*}
\sum\limits_{t\in T}f_{n+a_{t}}  &  =\sum\limits_{s\in S}f_{n+s}%
\ \ \ \ \ \ \ \ \ \ \left(  \text{by the condition of Lemma~\ref{lem.9}%
}\right) \\
&  =\sum\limits_{t\in S+n}f_{t}\ \ \ \ \ \ \ \ \ \ \left(
\begin{array}
[c]{c}%
\text{here, we substituted }t\text{ for }n+s\text{, since the map}\\
S\rightarrow S+n,\ s\mapsto n+s\text{ is a bijection}%
\end{array}
\right)  .
\end{align*}
Similarly, $\sum\limits_{t\in T}f_{n+a_{t}}=\sum\limits_{t\in S^{\prime}%
+n}f_{t}$. Hence, $\sum\limits_{t\in S+n}f_{t}=\sum\limits_{t\in S^{\prime}%
+n}f_{t}$. Since the sets $S+n$ and $S^{\prime}+n$ are both lacunar (since so
are $S$ and $S^{\prime}$) and finite (since so are $S$ and $S^{\prime}$), and
are subsets of $\mathbb{N}_{2}$ (here is where we use (\ref{l7})), we can now
conclude from Lemma~\ref{lem.4} that $S+n=S^{\prime}+n$, so that $S=S^{\prime
}$, proving Lemma~\ref{lem.9}.
\end{proof}

\begin{proof}
[Proof of Theorem~\ref{thm.1}.]Now, Theorem~\ref{thm.1} is clear, since the
existence follows from Theorem~\ref{thm.8} and the uniqueness from
Lemma~\ref{lem.9}.
\end{proof}

\begin{thebibliography}{9}                                                                                                %


\bibitem[1]{1}Philip Matchett Wood, Doron Zeilberger, \textit{A translation
method for finding combinatorial bijections}, Annals of Combinatorics
\textbf{13} (2009), pp. 383--402. \newline\url{http://www.math.rutgers.edu/~zeilberg/mamarim/mamarimhtml/trans-method.html}

\bibitem[2]{this.long}Darij Grinberg, \textit{Zeckendorf family identities
generalized},
%TCIMACRO{\TeXButton{today}{\today}}%
%BeginExpansion
\today
%EndExpansion
, *long version*.\newline%
\url{http://www.cip.ifi.lmu.de/~grinberg/zeckendorfLONG.pdf}\newline Also
available as an ancillary file to
\href{https://arxiv.org/abs/1103.4507v2}{arXiv preprint arXiv:1103.4507v2}.
\end{thebibliography}


\end{document}