\documentclass[12pt,final,notitlepage,onecolumn]{article}%
\usepackage{amsfonts}
\usepackage{amssymb}
\usepackage{graphicx}
\usepackage{amsmath}
\usepackage[breaklinks=true]{hyperref}%
\setcounter{MaxMatrixCols}{30}
\voffset=-2.5cm
\hoffset=-2.5cm
\setlength\textheight{24cm}
\setlength\textwidth{15.5cm}
\begin{document}

%NOT proofreading completed


\title{Zeckendorf family identities generalized}
\author{Darij Grinberg}
\date{22 March 2011, brief version}
\maketitle

\begin{abstract}
In [1], Philip Matchett Wood and Doron Zeilberger have constructed identities
for the Fibonacci numbers $f_{n}$ of the form%
\begin{align*}
1f_{n} &  =f_{n}\text{ for all }n\geq1;\\
2f_{n} &  =f_{n-2}+f_{n+1}\text{ for all }n\geq3;\\
3f_{n} &  =f_{n-2}+f_{n+2}\text{ for all }n\geq3;\\
4f_{n} &  =f_{n-2}+f_{n}+f_{n+2}\text{ for all }n\geq3;\\
&  \text{etc.;}\\
kf_{n} &  =\sum_{s\in S_{k}}f_{n+s}\text{ for all }n>\max\left\{  -s\mid s\in
S_{k}\right\}  \text{,}%
\end{align*}
where $S_{k}$ is a fixed ``lacunar'' set of integers (``lacunar'' means that no two
elements of this set are consecutive integers) depending only on $k$. (The
condition $n>\max\left\{  -s\mid s\in S_{k}\right\}  $ is only to make sure
that all addends $f_{n+s}$ are well-defined. If the Fibonacci sequence is
properly continued to the negative, this condition drops out.)\newline In this
note we prove a generalization of these identities: For any family $\left(
a_{1},a_{2},...,a_{p}\right)  $ of integers, there exists one and only one
finite lacunar set $S$ of integers such that every $n$ (high enough to make the
Fibonacci numbers in the equation below well-defined) satisfies
\[
f_{n+a_{1}}+f_{n+a_{2}}+...+f_{n+a_{p}}=\sum\limits_{s\in S}f_{n+s}.
\]
The proof uses the Fibonacci-approximating properties of the golden ratio.
It would be interesting to find a purely combinatorial proof.
\end{abstract}

%

\begin{tabular}
[c]{||l||}\hline\hline
This is a brief version of my text [2]. For more detailed proofs, see [2] (but
beware that [2] is\\
sometimes over-precise and very boring).\\
This note has never been proofread by myself or anyone else. If you find any
mistakes or typos,\\
please inform me at $\Delta\Gamma$\texttt{@gmail.com} where $\Delta
=$\texttt{darij }and $\Gamma=$\texttt{grinberg}\\
Thank you!\\\hline\hline
\end{tabular}


\bigskip

The purpose of this note is to establish a generalization of the so-called
\textit{Zeckendorf family identities} which were discussed in [1]. First, some definitions:

\begin{quote}
\textbf{Definitions.} \textbf{1)} A subset $S$ of $\mathbb{Z}$ is called
\textit{lacunar} if it satisfies $\left(  s+1\notin S\text{ for every }s\in
S\right)  $.

\textbf{2)} Let $\left(  f_{1},f_{2},f_{3},...\right)  $ be the Fibonacci
sequence (defined by $f_{1}=f_{2}=1$ and the recurrence relation $\left(
f_{n}=f_{n-1}+f_{n-2}\text{ for all }n\in\mathbb{N}\text{ satisfying }%
n\geq3\right)  $).
\end{quote}

Our main theorem is:

\begin{quote}
\textbf{Theorem 1 (generalized Zeckendorf family identities).} Let $T$ be a
finite set, and $a_{t}$ be an integer for every $t\in T$.

Then, there exists one and only one finite lacunar subset $S$ of $\mathbb{Z}$
such that
\[
\left(  \sum\limits_{t\in T}f_{n+a_{t}}=\sum\limits_{s\in S}f_{n+s}\text{ for
every }n\in\mathbb{Z}\text{ which satisfies }n>\max\left(  \left\{  -a_{t}\mid
t\in T\right\}  \cup\left\{  -s\mid s\in S\right\}  \right)  \right)  .
\]



\end{quote}

\textbf{Remarks.}

\textbf{1)} The \textit{Zeckendorf family identities}\footnote{The first seven
of these identities are%
\begin{align*}
1f_{n} &  =f_{n}\text{ for all }n\geq1;\\
2f_{n} &  =f_{n-2}+f_{n+1}\text{ for all }n\geq3;\\
3f_{n} &  =f_{n-2}+f_{n+2}\text{ for all }n\geq3;\\
4f_{n} &  =f_{n-2}+f_{n}+f_{n+2}\text{ for all }n\geq3;\\
5f_{n} &  =f_{n-4}+f_{n-1}+f_{n+3}\text{ for all }n\geq5;\\
6f_{n} &  =f_{n-4}+f_{n+1}+f_{n+3}\text{ for all }n\geq5;\\
7f_{n} &  =f_{n-4}+f_{n+4}\text{ for all }n\geq5.
\end{align*}
} from [1] are the result of applying Theorem 1 to the case when all $a_{t}$
are $=0$.

\textbf{2)} The condition $n>\max\left(  \left\{  -a_{t}\mid t\in T\right\}
\cup\left\{  -s\mid s\in S\right\}  \right)  $ in Theorem 1 is just a
technical condition made in order to ensure that the Fibonacci numbers
$f_{n+a_{t}}$ for all $t\in T$ and $f_{n+s}$ for all $s\in S$ are
well-defined. (If we would define the Fibonacci numbers $f_{n}$ for integers
$n\leq0$ by extending the recurrence relation $f_{n}=f_{n-1}+f_{n-2}$ ``to the
left'', then we could drop this condition.)

The following is my proof of Theorem 1. It does not even try to be
combinatorial - it is pretty much the opposite. Technically, it is completely
elementary and does not resort to any theorems from analysis; but the method
used (choosing a ``large enough'' $N$ to make an estimate work) is an analytic one.

First, some lemmas and notations:

We denote by $\mathbb{N}$ the set $\left\{  0,1,2,...\right\}  $ (and not the
set $\left\{  1,2,3,...\right\}  $, like some other authors do). Also, we
denote by $\mathbb{N}_{2}$ the set $\left\{  2,3,4,...\right\}  =\mathbb{N}%
\setminus\left\{  0,1\right\}  $.

Also, let $\phi=\dfrac{1+\sqrt{5}}{2}$. We notice that $\phi\approx1.618...$
and that $\phi^{2}=\phi+1$.

First, some basic (and known) lemmas on the Fibonacci sequence:

\begin{quote}
\textbf{Lemma 2.} If $S$ is a finite lacunar subset of $\mathbb{N}_{2}$, then
$\sum\limits_{t\in S}f_{t}<f_{\max S+1}$.
\end{quote}

\textit{Proof.} This is rather clear either by a telescoping sum argument
(write the set $S$ in the form $\left\{  s_{1},s_{2},...,s_{k}\right\}  $ with
$s_{1}<s_{2}<...<s_{k}$, notice that
\[
\sum\limits_{t\in S}f_{t}=\sum\limits_{i=1}^{k}f_{s_{i}}=\sum\limits_{i=1}%
^{k}\left(  f_{s_{i}+1}-f_{s_{i}-1}\right)  =\sum\limits_{i=1}^{k-1}\left(
f_{s_{i}+1}-f_{s_{i+1}-1}\right)  +\underbrace{f_{s_{k}+1}}_{=f_{\max S+1}%
}-\underbrace{f_{s_{1}-1}}_{>0},
\]
and use $s_{i}+1\leq s_{i+1}-1$ since the set $S$ is lacunar) or by induction
over $\max S$ (use $f_{\max S+1}=f_{\max S}+f_{\max S-1}$ here).

\begin{quote}
\textbf{Lemma 3 (existence part of the Zeckendorf theorem).} For every
nonnegative integer $n$, there exists a finite lacunar subset $T$ of
$\mathbb{N}_{2}$ such that $n=\sum\limits_{t\in T}f_{t}$.
\end{quote}

\textit{Proof.} Induction over $n$. The main idea here is to let $t_{1}$ be
the maximal $\tau\in\mathbb{N}_{2}$ satisfying $f_{\tau}\leq n$, and apply
Lemma 3 to $n-t_{1}$ instead of $n$. The details are left to the reader (and
can be found in [2]).

\begin{quote}
\textbf{Lemma 4 (uniqueness part of the Zeckendorf theorem).} Let
$n\in\mathbb{N}$, and let $T$ and $T^{\prime}$ be two finite lacunar subsets of
$\mathbb{N}_{2}$ such that $n=\sum\limits_{t\in T}f_{t}$ and $n=\sum
\limits_{t\in T^{\prime}}f_{t}$. Then, $T=T^{\prime}$.
\end{quote}

\textit{Proof.} Induction over $n$. Use Lemma 2 to show that $\max T<\max
T^{\prime}+1$ and $\max T^{\prime}<\max T+1$, resulting in $\max T=\max
T^{\prime}$. Hence, the sets $T$ and $T^{\prime}$ have an element in common,
and we can reduce the situation to one with a smaller $n$ by removing this
common element from both sets.

Lemmata 3 and 4 together yield:

\begin{quote}
\textbf{Theorem 5 (Zeckendorf theorem).} For every nonnegative integer $n$,
there exists one and only one finite lacunar subset $T$ of $\mathbb{N}_{2}$ such
that $n=\sum\limits_{t\in T}f_{t}$.

We will denote this set $T$ by $Z_{n}$. Thus, $n=\sum\limits_{t\in Z_{n}}%
f_{t}$.
\end{quote}

Now for something completely trivial:

\begin{quote}
\textbf{Theorem 6.} For every $n\in\mathbb{N}_{2}$, we have $\left\vert
f_{n+1}-\phi f_{n}\right\vert =\dfrac{1}{\sqrt{5}}\left(  \phi-1\right)  ^{n}$.
\end{quote}

\textit{Proof.} Binet's formula yields $f_{n}=\dfrac{\phi^{n}-\phi^{-n}}%
{\sqrt{5}}$ and $f_{n+1}=\dfrac{\phi^{n+1}-\phi^{-\left(  n+1\right)  }}%
{\sqrt{5}}$; the rest is computation.

Yet another lemma:

\begin{quote}
\textbf{Theorem 7.} If $S$ is a finite lacunar subset of $\mathbb{N}_{2}$, then
$\sum\limits_{s\in S}\left(  \phi-1\right)  ^{s}\leq\phi-1$.
\end{quote}

\textit{Proof of Theorem 7.} Since $S$ is a lacunar subset of $\mathbb{N}_{2}$,
the smallest element of $S$ is at least $2$, the second smallest element of
$S$ is at least $4$ (since it is larger than the smallest element by at least
$2$), the third smallest element of $S$ is at least $6$ (since it is larger
than the second smallest element by at least $2$), and so on. Since
$\mathbb{N}\rightarrow\mathbb{R}$, $s\mapsto\left(  \phi-1\right)  ^{s}$ is a
monotonically decreasing function (as $0\leq\phi-1\leq1$), we thus have%
\[
\sum_{s\in S}\left(  \phi-1\right)  ^{s}\leq\sum_{s\in\left\{
2,4,6,...\right\}  }\left(  \phi-1\right)  ^{s}=\sum_{t\in\left\{
1,2,3,...\right\}  }\left(  \phi-1\right)  ^{2t}=\phi-1
\]
(by the formula for the sum of the geometric series, along with some
computations). This proves Theorem 7.

Let us now come to the proof of Theorem 1. First, we formulate the existence
part of this theorem:

\begin{quote}
\textbf{Theorem 8 (existence part of the generalized Zeckendorf family
identities).} Let $T$ be a finite set, and $a_{t}$ be an integer for every
$t\in T$.

Then, there exists a finite lacunar subset $S$ of $\mathbb{Z}$ such that
\[
\left(  \sum\limits_{t\in T}f_{n+a_{t}}=\sum\limits_{s\in S}f_{n+s}\text{ for
every }n\in\mathbb{Z}\text{ which satisfies }n>\max\left(  \left\{  -a_{t}\mid
t\in T\right\}  \cup\left\{  -s\mid s\in S\right\}  \right)  \right)  .
\]



\end{quote}

Before we start proving this, we need a new notation:

\begin{quote}
\textbf{Definition.} Let $K$ be a subset of $\mathbb{Z}$, and $a\in\mathbb{Z}%
$. Then, $K+a$ will denote the subset $\left\{  k+a\ \mid\ k\in K\right\}  $
of $\mathbb{Z}$.
\end{quote}

Clearly, $\left(  K+a\right)  +b=K+\left(  a+b\right)  $ for any two integers
$a$ and $b$. Also, $K+0=K$. Finally, if $K$ is a lacunar subset of $\mathbb{Z}$,
and if $a\in\mathbb{Z}$, then $K+a$ is lacunar as well.

\textit{Proof of Theorem 8.} Choose a high enough integer $N$. What exactly
``high enough'' means we will see later; at the moment, we only require
$N\in\mathbb{N}_{2}$ and $N>\max\left\{  -a_{t}\mid t\in T\right\}  $. We
might later want $N$ to be even higher, however.

Let $\nu=\sum\limits_{t\in T}f_{N+a_{t}}$. Then, Lemma 3 yields $\nu
=\sum\limits_{t\in Z_{\nu}}f_{t}$ for a finite lacunar subset $Z_{\nu}$ of
$\mathbb{N}_{2}$. Let $S=\left\{  t-N\ \mid\ t\in Z_{\nu}\right\}  $. Then,
$S=Z_{\nu}+\left(  -N\right)  $ is a finite lacunar subset of $\mathbb{Z}$, and
$\nu=\sum\limits_{t\in Z_{\nu}}f_{t}$ becomes $\nu=\sum\limits_{s\in S}%
f_{N+s}$. So now we know that $\sum\limits_{t\in T}f_{N+a_{t}}=\sum
\limits_{s\in S}f_{N+s}$ (because both sides of this equation equal $\nu$).

So, we have chosen a high $N$ and found a finite lacunar subset $S$ of
$\mathbb{Z}$ which satisfies $\sum\limits_{t\in T}f_{N+a_{t}}=\sum
\limits_{s\in S}f_{N+s}$. But Theorem 8 is not proven yet: Theorem 8 requires
us to prove that there exists \textit{one} finite lacunar subset $S$ of
$\mathbb{Z}$ which works for \textit{every} $N$, while at the moment we cannot
be sure yet whether different $N$'s wouldn't produce \textit{different} sets
$S$. And, in fact, different $N$'s \textit{can} produce different sets $S$,
but (fortunately!) only if the $N$'s are too small. If we take $N$ high
enough, the set $S$ that we obtained turns out to be \textit{universal}, i. e.
it satisfies
\begin{equation}
\sum\limits_{t\in T}f_{n+a_{t}}=\sum\limits_{s\in S}f_{n+s}%
\ \ \ \ \ \ \ \ \ \ \text{for every }n\in\mathbb{Z}\text{ which satisfies
}n>\max\left(  \left\{  -a_{t}\mid t\in T\right\}  \cup\left\{  -s\mid s\in
S\right\}  \right)  . \label{BigLemma}%
\end{equation}
We are now going to prove this.

In order to prove (\ref{BigLemma}), we need two assertions:

\textit{Assertion 1:} If some $n\in\mathbb{Z}$ satisfies $n\geq N$ and
$\sum\limits_{t\in T}f_{n+a_{t}}=\sum\limits_{s\in S}f_{n+s}$, then
$\sum\limits_{t\in T}f_{\left(  n+1\right)  +a_{t}}=\sum\limits_{s\in
S}f_{\left(  n+1\right)  +s}$.

\textit{Assertion 2:} If some $n\in\mathbb{Z}$ satisfies $\sum\limits_{t\in
T}f_{n+a_{t}}=\sum\limits_{s\in S}f_{n+s}$ and $\sum\limits_{t\in T}f_{\left(
n+1\right)  +a_{t}}=\sum\limits_{s\in S}f_{\left(  n+1\right)  +s}$, then
$\sum\limits_{t\in T}f_{\left(  n-1\right)  +a_{t}}=\sum\limits_{s\in
S}f_{\left(  n-1\right)  +s}$ (if $n-1>\max\left(  \left\{  -a_{t}\mid t\in
T\right\}  \cup\left\{  -s\mid s\in S\right\}  \right)  $).

Obviously, Assertion 1 yields (by induction) that $\sum\limits_{t\in
T}f_{n+a_{t}}=\sum\limits_{s\in S}f_{n+s}$ for every $n\geq N$, and Assertion
2 then finishes off the remaining $n$'s (by backwards induction, or, to be
more precise, by an induction step from $n+1$ and $n$ to $n-1$). Thus, once
both Assertions 1 and 2 are proven, (\ref{BigLemma}) will follow and thus
Theorem 8 will be proven.

Assertion 2 is almost trivial (just notice that%
\[
\sum\limits_{t\in T}f_{\left(  n-1\right)  +a_{t}}=\sum\limits_{t\in
T}\underbrace{f_{n+a_{t}-1}}_{=f_{n+a_{t}+1}-f_{n+a_{t}}}=\sum\limits_{t\in
T}f_{n+a_{t}+1}-\sum\limits_{t\in T}f_{n+a_{t}}=\sum\limits_{t\in T}f_{\left(
n+1\right)  +a_{t}}-\sum\limits_{t\in T}f_{n+a_{t}}%
\]
and%
\[
\sum\limits_{s\in S}f_{\left(  n-1\right)  +s}=\sum\limits_{s\in
S}\underbrace{f_{n+s-1}}_{=f_{n+s+1}-f_{n+s}}=\sum\limits_{s\in S}%
f_{n+s+1}-\sum\limits_{s\in S}f_{n+s}=\sum\limits_{s\in S}f_{\left(
n+1\right)  +s}-\sum\limits_{s\in S}f_{n+s}%
\]
), so it only remains to prove Assertion 1.

So let us prove Assertion 1. Here we are going to use that $N$ is high enough
(because otherwise, Assertion 1 wouldn't hold). We have $\sum\limits_{t\in
T}f_{n+a_{t}}=\sum\limits_{s\in S}f_{n+s}$ by assumption, so that
$\sum\limits_{t\in T}f_{n+a_{t}}-\sum\limits_{s\in S}f_{n+s}=0$. Thus,%
\begin{align*}
\sum\limits_{t\in T}f_{\left(  n+1\right)  +a_{t}}-\sum\limits_{s\in
S}f_{\left(  n+1\right)  +s}  &  =\sum\limits_{t\in T}f_{\left(  n+1\right)
+a_{t}}-\sum\limits_{s\in S}f_{\left(  n+1\right)  +s}-\phi\left(
\sum\limits_{t\in T}f_{n+a_{t}}-\sum\limits_{s\in S}f_{n+s}\right) \\
&  =\sum\limits_{t\in T}\left(  f_{\left(  n+1\right)  +a_{t}}-\phi
f_{n+a_{t}}\right)  -\sum\limits_{s\in S}\left(  f_{\left(  n+1\right)
+s}-\phi f_{n+s}\right) \\
&  =\sum\limits_{t\in T}\left(  f_{n+a_{t}+1}-\phi f_{n+a_{t}}\right)
-\sum\limits_{s\in S}\left(  f_{n+s+1}-\phi f_{n+s}\right)  ,
\end{align*}
so that%
\begin{align}
&  \left\vert \sum\limits_{t\in T}f_{\left(  n+1\right)  +a_{t}}%
-\sum\limits_{s\in S}f_{\left(  n+1\right)  +s}\right\vert =\left\vert
\sum\limits_{t\in T}\left(  f_{n+a_{t}+1}-\phi f_{n+a_{t}}\right)
-\sum\limits_{s\in S}\left(  f_{n+s+1}-\phi f_{n+s}\right)  \right\vert
\nonumber\\
&  \leq\sum\limits_{t\in T}\left\vert f_{n+a_{t}+1}-\phi f_{n+a_{t}%
}\right\vert +\sum\limits_{s\in S}\left\vert f_{n+s+1}-\phi f_{n+s}\right\vert
\ \ \ \ \ \ \ \ \ \ \left(  \text{by the triangle inequality}\right)
\nonumber\\
&  =\dfrac{1}{\sqrt{5}}\sum\limits_{t\in T}\left(  \phi-1\right)  ^{n+a_{t}%
}+\dfrac{1}{\sqrt{5}}\sum\limits_{s\in S}\left(  \phi-1\right)  ^{n+s}%
\ \ \ \ \ \ \ \ \ \ \left(  \text{by Theorem 6}\right) \nonumber\\
&  <\sum\limits_{t\in T}\left(  \phi-1\right)  ^{n+a_{t}}+\sum\limits_{s\in
S}\left(  \phi-1\right)  ^{n+s}\ \ \ \ \ \ \ \ \ \ \left(  \text{since }%
\dfrac{1}{\sqrt{5}}<1\right) \nonumber\\
&  \leq\sum\limits_{t\in T}\left(  \phi-1\right)  ^{N+a_{t}}+\sum\limits_{s\in
S}\left(  \phi-1\right)  ^{N+s}\ \ \ \ \ \ \ \ \ \ \left(
\begin{array}
[c]{c}%
\text{since }\left(  \phi-1\right)  ^{n+a_{t}}\leq\left(  \phi-1\right)
^{N+a_{t}}\text{ and}\\
\left(  \phi-1\right)  ^{n+s}\leq\left(  \phi-1\right)  ^{N+s}\text{,
because}\\
n\geq N\text{ and }0\leq\phi-1\leq1
\end{array}
\right) \nonumber\\
&  =\sum\limits_{t\in T}\left(  \phi-1\right)  ^{N+a_{t}}+\sum\limits_{t\in
Z_{\nu}}\left(  \phi-1\right)  ^{t}\ \ \ \ \ \ \ \ \ \ \left(  \text{since
}S=\left\{  t-N\ \mid\ t\in Z_{\nu}\right\}  \right) \nonumber\\
&  =\sum\limits_{t\in T}\left(  \phi-1\right)  ^{N+a_{t}}+\sum\limits_{s\in
Z_{\nu}}\left(  \phi-1\right)  ^{s}=\left(  \phi-1\right)  ^{N}\sum
\limits_{t\in T}\left(  \phi-1\right)  ^{a_{t}}+\sum\limits_{s\in Z_{\nu}%
}\left(  \phi-1\right)  ^{s}\nonumber\\
&  \leq\left(  \phi-1\right)  ^{N}\sum\limits_{t\in T}\left(  \phi-1\right)
^{a_{t}}+\left(  \phi-1\right)  \label{Estimate}%
\end{align}
(since $\sum\limits_{s\in Z_{\nu}}\left(  \phi-1\right)  ^{s}\leq\phi-1$ by
Theorem 7, because $Z_{\nu}$ is a lacunar subset of $\mathbb{N}_{2}$).

Now, $\sum\limits_{t\in T}\left(  \phi-1\right)  ^{a_{t}}$ is a constant,
while $\left(  \phi-1\right)  ^{N}\rightarrow0$ for $N\rightarrow\infty$.
Hence, we can make the product $\left(  \phi-1\right)  ^{N}\sum\limits_{t\in
T}\left(  \phi-1\right)  ^{a_{t}}$ arbitrarily close to $0$ if we choose $N$
high enough. Since $\phi-1<1$, we have%
\begin{equation}
\left(  \phi-1\right)  ^{N}\sum\limits_{t\in T}\left(  \phi-1\right)  ^{a_{t}%
}+\left(  \phi-1\right)  <1 \label{BoundWIN}%
\end{equation}
if $\left(  \phi-1\right)  ^{N}\sum\limits_{t\in T}\left(  \phi-1\right)
^{a_{t}}$ is sufficiently close to $0$, what we can enforce by taking a high
enough $N$. This is exactly the point where we require $N$ to be high enough.

So let us take $N$ high enough so that (\ref{BoundWIN}) holds. Combined with
(\ref{Estimate}), it then yields%
\[
\left\vert \sum\limits_{t\in T}f_{\left(  n+1\right)  +a_{t}}-\sum
\limits_{s\in S}f_{\left(  n+1\right)  +s}\right\vert <1,
\]
which leads to $\left\vert \sum\limits_{t\in T}f_{\left(  n+1\right)  +a_{t}%
}-\sum\limits_{s\in S}f_{\left(  n+1\right)  +s}\right\vert =0$ (since
$\left\vert \sum\limits_{t\in T}f_{\left(  n+1\right)  +a_{t}}-\sum
\limits_{s\in S}f_{\left(  n+1\right)  +s}\right\vert $ is a nonnegative
integer). In other words, $\sum\limits_{t\in T}f_{\left(  n+1\right)  +a_{t}%
}=\sum\limits_{s\in S}f_{\left(  n+1\right)  +s}$. This completes the proof of
Assertion 1, and, with it, the proof of Theorem 8.

All that remains now is the (rather trivial) uniqueness part of Theorem 1:

\begin{quote}
\textbf{Lemma 9 (uniqueness part of the generalized Zeckendorf family
identities).} Let $T$ be a finite set, and $a_{t}$ be an integer for every
$t\in T$.

Let $S$ be a finite lacunar subset of $\mathbb{Z}$ such that%
\[
\left(  \sum\limits_{t\in T}f_{n+a_{t}}=\sum\limits_{s\in S}f_{n+s}\text{ for
every }n\in\mathbb{Z}\text{ which satisfies }n>\max\left(  \left\{  -a_{t}\mid
t\in T\right\}  \cup\left\{  -s\mid s\in S\right\}  \right)  \right)  .
\]
Let $S^{\prime}$ be a finite lacunar subset of $\mathbb{Z}$ such that%
\[
\left(  \sum\limits_{t\in T}f_{n+a_{t}}=\sum\limits_{s\in S^{\prime}}%
f_{n+s}\text{ for every }n\in\mathbb{Z}\text{ which satisfies }n>\max\left(
\left\{  -a_{t}\mid t\in T\right\}  \cup\left\{  -s\mid s\in S^{\prime
}\right\}  \right)  \right)  .
\]
Then, $S=S^{\prime}$.
\end{quote}

\textit{Proof of Lemma 9.} Let%
\begin{equation}
n=\max\left(  \left\{  -a_{t}\mid t\in T\right\}  \cup\left\{  -s\mid s\in
S\right\}  \cup\left\{  -s\mid s\in S^{\prime}\right\}  \right)  +2.
\label{l7}%
\end{equation}
Then, $n$ satisfies $n>\max\left(  \left\{  -a_{t}\mid t\in T\right\}
\cup\left\{  -s\mid s\in S\right\}  \right)  $, so that
\begin{align*}
\sum\limits_{t\in T}f_{n+a_{t}}  &  =\sum\limits_{s\in S}f_{n+s}%
\ \ \ \ \ \ \ \ \ \ \left(  \text{by the condition of Lemma 9}\right) \\
&  =\sum\limits_{t\in S+n}f_{t}\ \ \ \ \ \ \ \ \ \ \left(
\begin{array}
[c]{c}%
\text{here, we substituted }t\text{ for }n+s\text{, since the map}\\
S\rightarrow S+n,\ s\mapsto n+s\text{ is a bijection}%
\end{array}
\right)  .
\end{align*}
Similarly, $\sum\limits_{t\in T}f_{n+a_{t}}=\sum\limits_{t\in S^{\prime}%
+n}f_{t}$. Hence, $\sum\limits_{t\in S+n}f_{t}=\sum\limits_{t\in S^{\prime}%
+n}f_{t}$. Since the sets $S+n$ and $S^{\prime}+n$ are both lacunar (since so
are $S$ and $S^{\prime}$) and finite (since so are $S$ and $S^{\prime}$), and
are subsets of $\mathbb{N}_{2}$ (here is where we use (\ref{l7})), we can now
conclude from Lemma 4 that $S+n=S^{\prime}+n$, so that $S=S^{\prime}$, proving
Lemma 9.

Now, Theorem 1 is clear, since the existence follows from\ Theorem 8 and the
uniqueness from Lemma 9.

\begin{center}
\textbf{References}
\end{center}

[1] Philip Matchett Wood, Doron Zeilberger, \textit{A translation method for
finding combinatorial bijections}, to appear in Annals of
Combinatorics.\newline\url{http://www.math.rutgers.edu/~zeilberg/mamarim/mamarimhtml/trans-method.html}

[2] Darij Grinberg, \textit{Zeckendorf family identities generalized}, version
7, 22 March 2011 *long version*.\newline
\url{http://www.cip.ifi.lmu.de/~grinberg/zeckendorfLONG.pdf}


\end{document}