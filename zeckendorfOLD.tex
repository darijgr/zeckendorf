\documentclass[12pt,final,notitlepage,onecolumn]{article}%
\usepackage{amsfonts}
\usepackage{amssymb}
\usepackage{graphicx}
\usepackage{amsmath}%
\setcounter{MaxMatrixCols}{30}
\voffset=-2.5cm
\hoffset=-2.5cm
\setlength\textheight{24cm}
\setlength\textwidth{15.5cm}
\begin{document}

%NOT proofreading completed
%not even trying to be a completely-written solution


\begin{center}
\textbf{Zeckendorf family identities generalized}

\textbf{Darij Grinberg, version 3, 20 September 2010}
\end{center}

A subset $S$ of $\mathbb{Z}$ is called \textit{holey} if it satisfies
$a+1\notin S$ for every $a\in S$.

Let $\left(  f_{1},f_{2},f_{3},...\right)  $ be the Fibonacci sequence
(defined by $f_{1}=f_{2}=1$ and the recurrence relation $\left(  f_{n}%
=f_{n-1}+f_{n-2}\text{ for all }n\in\mathbb{N}\text{ satisfying }%
n\geq3\right)  $).

Let $T$ be a finite set, and $a_{t}$ be an integer for every $t\in T$.

Prove that there exists one and only one holey subset $S$ of $\mathbb{Z}$ such
that
\[
\left(  \sum\limits_{t\in T}f_{n+a_{t}}=\sum\limits_{s\in S}f_{n+s}\text{ for
every }n\in\mathbb{Z}\text{ which satisfies }n>\max\left(  \left\{  -a_{t}\mid
t\in T\right\}  \cup\left\{  -s\mid s\in S\right\}  \right)  \right)  .
\]


\bigskip

\textbf{Remark.} This problem generalizes the so-called \textit{Zeckendorf
family identities} (which correspond to the case when all $a_{t}$ are equal),
which were discussed in [1].

\begin{center}
\textbf{Outline of a solution}
\end{center}

We denote by $\mathbb{N}$ the set $\left\{  0,1,2,...\right\}  $. Also, we
denote by $\mathbb{N}_{2}$ the set $\left\{  2,3,4,...\right\}  =\mathbb{N}%
\setminus\left\{  0,1\right\}  $.

Also, let $\phi=\dfrac{1+\sqrt{5}}{2}$. We notice that $\phi\approx1.618...$.

We recall some known facts about the Fibonacci sequence:

\textbf{Lemma 0.} [...]

\begin{quote}
\textbf{Theorem 1 (Zeckendorf theorem).} For every nonnegative integer $n$,
there exists one and only one finite holey subset $T$ of $\mathbb{N}_{2}$ such
that $n=\sum\limits_{t\in T}f_{t}$.

We will denote this set $T$ by $Z_{n}$. Thus, $n=\sum\limits_{t\in Z_{n}}%
f_{t}$.
\end{quote}

\textit{Proof of Theorem 1 (sketched).} \textit{Existence of }$T$\textit{:}
Induction on $n$. For $n=0,$ take $T=\varnothing$. Else, let $t_{1}$ be the
maximal integer from $\mathbb{N}_{2}$ satisfying $f_{t_{1}}\leq n$, and apply
the induction hypothesis to $n-f_{t_{1}}$ instead of $n$. So you get a finite
holey subset $T^{\prime}$ of $\mathbb{N}_{2}$ such that $n-f_{t_{1}}%
=\sum\limits_{t\in T^{\prime}}f_{t}$. Thus, $n=\sum\limits_{t\in T}f_{t}$,
where $T=T^{\prime}\cup\left\{  t_{1}\right\}  $. The only thing remaining to
be checked is that this new set $T=T^{\prime}\cup\left\{  t_{1}\right\}  $ is
holey. But in fact, every element of $T^{\prime}$ is $<t_{1}-1$ (since
otherwise, the sum $\sum\limits_{t\in T^{\prime}}f_{t}$ would contain an
addend greater or equal to $f_{t_{1}-1}$, so that$\ n-f_{t_{1}}=\sum
\limits_{t\in T^{\prime}}f_{t}\geq f_{t_{1}-1}$, and thus $n\geq f_{t_{1}%
}+f_{t_{1}-1}=f_{t_{1}+1}$, contradicting to the maximality of $t_{1}$). So
the existence of $T$ is proven.

\textit{Uniqueness of }$T$\textit{:} Assume that there are two holey subsets
$T$ and $T^{\prime}$ of $\mathbb{N}_{2}$ such that $\sum\limits_{t\in T}%
f_{t}=\sum\limits_{t\in T^{\prime}}f_{t}$. We want to prove that $T=T^{\prime
}$. In fact, assume that $\max T>\max T^{\prime}$ (in fact, if $\max T<\max
T^{\prime}$, then we can just permute $T$ and $T^{\prime}$, and if $\max
T=\max T^{\prime}$, then we can replace the sets $T$ and $T^{\prime}$ by the
smaller sets $T\setminus\left\{  \max T\right\}  $ and $T^{\prime}%
\setminus\left\{  \max T^{\prime}\right\}  $). Then, $\max T\geq\max
T^{\prime}+1$, so that $\sum\limits_{t\in T^{\prime}}f_{t}=\sum\limits_{t\in
T}f_{t}\geq f_{\max T}\geq f_{\max T^{\prime}+1}$. Now, in order to obtain a
contradiction, we need a lemma: If $S$ is a holey subset of $\mathbb{N}_{2}$,
then $\sum\limits_{t\in S}f_{t}<f_{\max S+1}$. To prove this lemma, notice
that $\sum\limits_{t\in S}f_{t}<f_{\max S+1}$ is equivalent to $\sum
\limits_{t\in S\setminus\left\{  \max S\right\}  }f_{t}<f_{\max S-1}$ (because
$\sum\limits_{t\in S\setminus\left\{  \max S\right\}  }f_{t}=\sum\limits_{t\in
S}f_{t}-f_{\max S}$ and $f_{\max S-1}=f_{\max S+1}-f_{\max S}$), which in turn
follows from the lemma itself applied to $S\setminus\left\{  \max S\right\}  $
instead of $S$ (so we get an induction step), because $\max S-1\geq\max\left(
S\setminus\left\{  \max S\right\}  \right)  +1$ (here, we use the holeyness of
$S$).

Altogether, Theorem 1 is proven. Several books should have this proof in more detail.

\begin{quote}
\textbf{Theorem 2 (rounded Binet recurrence).} For every $n\in\mathbb{N}_{2}$,
we have $\left\vert f_{n+1}-\phi f_{n}\right\vert =\left(  \phi-1\right)
^{n}\leq\left(  \phi-1\right)  ^{2}<\dfrac{1}{2}$.
\end{quote}

\textit{Proof of Theorem 2.} By Binet's formula,%
\[
f_{n}=\dfrac{\phi^{n}-\left(  1-\phi\right)  ^{n}}{\sqrt{5}}%
\ \ \ \ \ \ \ \ \ \ \text{and}\ \ \ \ \ \ \ \ \ \ f_{n+1}=\dfrac{\phi
^{n+1}-\left(  1-\phi\right)  ^{n+1}}{\sqrt{5}},
\]
what results in%
\begin{align*}
\left\vert f_{n+1}-\phi f_{n}\right\vert  &  =\left\vert \dfrac{\phi\left(
1-\phi\right)  ^{n}-\left(  1-\phi\right)  ^{n+1}}{\sqrt{5}}\right\vert
=\left\vert \dfrac{\left(  2\phi-1\right)  \left(  1-\phi\right)  ^{n}}%
{\sqrt{5}}\right\vert =\dfrac{\left\vert 2\phi-1\right\vert \cdot\left\vert
1-\phi\right\vert ^{n}}{\sqrt{5}}\\
&  =\left\vert 1-\phi\right\vert ^{n}\ \ \ \ \ \ \ \ \ \ \left(  \text{since
}\left\vert 2\phi-1\right\vert =\sqrt{5}\right) \\
&  =\left(  \phi-1\right)  ^{n}\leq\left(  \phi-1\right)  ^{2}%
\ \ \ \ \ \ \ \ \ \ \left(  \text{since }0\leq\phi-1\leq1\text{ and }%
n\geq2\right) \\
&  <\dfrac{1}{2}\ \ \ \ \ \ \ \ \ \ \left(  \text{since }1=\left(
\phi-1\right)  ^{2}+\underbrace{\left(  \phi-1\right)  }_{\substack{>\left(
\phi-1\right)  ^{2},\\\text{since }0<\phi-1<1}}>2\left(  \phi-1\right)
^{2}\right)  ,
\end{align*}
and Theorem 2 is proven.

Yet another lemma:

\begin{quote}
\textbf{Theorem 3.} If $S$ is a holey subset of $\mathbb{N}_{2}$, then%
\[
\sum_{s\in S}\left(  \phi-1\right)  ^{s}\leq\phi-1<1.
\]



\end{quote}

\textit{Proof of Theorem 3.} Since $S$ is a holey subset of $\mathbb{N}_{2}$,
the smallest element of $S$ is at least $2$, the second smallest element of
$S$ is at least $4$ (since it is larger than the smallest element by at least
$2$), the third smallest element of $S$ is at least $6$ (since it is larger
than the second smallest element by at least $2$), and so on. Since
$\mathbb{N}\rightarrow\mathbb{R}$, $s\mapsto\left(  \phi-1\right)  ^{s}$ is a
monotonically decreasing function (as $0\leq\phi-1\leq1$), we thus have%
\begin{align*}
\sum_{s\in S}\left(  \phi-1\right)  ^{s}  &  \leq\sum_{s\in\left\{
2,4,6,...\right\}  }\left(  \phi-1\right)  ^{s}=\sum_{t\in\left\{
1,2,3,...\right\}  }\left(  \phi-1\right)  ^{2t}=\sum_{t=0}^{\infty}\left(
\phi-1\right)  ^{2t}-1\\
&  =\sum_{t=0}^{\infty}\left(  \left(  \phi-1\right)  ^{2}\right)
^{t}-1=\dfrac{1}{1-\left(  \phi-1\right)  ^{2}}-1=\dfrac{1}{\phi
-1}-1\ \ \ \ \ \ \ \ \ \ \left(  \text{since }1-\left(  \phi-1\right)
^{2}=\phi-1\right) \\
&  =\phi-1<1,
\end{align*}
and Theorem 3 is proven.

Let us now solve the problem.

Choose a large enough integer $N$. What exactly "large enough" means we will
see later; at the moment, we only require $N\in\mathbb{N}_{2}$ and
$N>\max\left\{  -a_{t}\mid t\in T\right\}  $. We will later want $N$ to be
even bigger, however.

Let $\nu=\sum\limits_{t\in T}f_{N+a_{t}}$. Then, Theorem 1 yields $\nu
=\sum\limits_{t\in Z_{\nu}}f_{t}$ for a finite holey subset $Z_{\nu}$ of
$\mathbb{N}_{2}$. Let $S=\left\{  t-N\ \mid\ t\in Z_{\nu}\right\}  $. Then,
$S$ is a finite holey subset of $\mathbb{Z}$, and $\nu=\sum\limits_{t\in
Z_{\nu}}f_{t}$ becomes $\nu=\sum\limits_{s\in S}f_{N+s}$. So now we know that
$\sum\limits_{t\in T}f_{N+a_{t}}=\sum\limits_{s\in S}f_{N+s}$ (because both
sides of this equation equal $\nu$).

So, we have chosen a large $N$ and found a finite holey subset $S$ of
$\mathbb{Z}$ which satisfies $\sum\limits_{t\in T}f_{N+a_{t}}=\sum
\limits_{s\in S}f_{N+s}$. But the problem is not solved yet: The problem
requires us to prove that there exists one finite holey subset $S$ of
$\mathbb{Z}$ which works for every $N$, while at the moment we cannot be sure
yet whether different $N$'s wouldn't produce different sets $S$. And, in fact,
different $N$'s can produce different sets $S$, but only if the $N$'s are too
small. If we take $N$ big enough, the set $S$ that we obtained turns out to be
\textit{universal}, i. e. it satisfies
\begin{equation}
\sum\limits_{t\in T}f_{n+a_{t}}=\sum\limits_{s\in S}f_{n+s}%
\ \ \ \ \ \ \ \ \ \ \text{for every }n\in\mathbb{Z}\text{ which satisfies
}n>\max\left(  \left\{  -a_{t}\mid t\in T\right\}  \cup\left\{  -s\mid s\in
S\right\}  \right)  . \label{BigLemma}%
\end{equation}
We are now going to prove this.

In order to prove (\ref{BigLemma}), we need two assertions:

\textit{Assertion 1:} If some $n\in\mathbb{Z}$ satisfies $n\geq N$ and
$\sum\limits_{t\in T}f_{n+a_{t}}=\sum\limits_{s\in S}f_{n+s}$, then
$\sum\limits_{t\in T}f_{\left(  n+1\right)  +a_{t}}=\sum\limits_{s\in
S}f_{\left(  n+1\right)  +s}$.

\textit{Assertion 2:} If some $n\in\mathbb{Z}$ satisfies $\sum\limits_{t\in
T}f_{n+a_{t}}=\sum\limits_{s\in S}f_{n+s}$ and $\sum\limits_{t\in T}f_{\left(
n+1\right)  +a_{t}}=\sum\limits_{s\in S}f_{\left(  n+1\right)  +s}$, then
$\sum\limits_{t\in T}f_{\left(  n-1\right)  +a_{t}}=\sum\limits_{s\in
S}f_{\left(  n-1\right)  +s}$ (if $n-1>\max\left(  \left\{  -a_{t}\mid t\in
T\right\}  \cup\left\{  -s\mid s\in S\right\}  \right)  $).

Obviously, Assertion 1 yields (by induction) that $\sum\limits_{t\in
T}f_{n+a_{t}}=\sum\limits_{s\in S}f_{n+s}$ for every $n\geq N$, and Assertion
2 then finishes off the remaining $n$'s (by backwards induction). Thus, once
both Assertions 1 and 2 are proven, (\ref{BigLemma}) will follow and thus the
problem will be solved.

Assertion 2 is almost trivial (just notice that%
\[
\sum\limits_{t\in T}f_{\left(  n-1\right)  +a_{t}}=\sum\limits_{t\in
T}\underbrace{f_{n+a_{t}-1}}_{=f_{n+a_{t}+1}-f_{n+a_{t}}}=\sum\limits_{t\in
T}f_{n+a_{t}+1}-\sum\limits_{t\in T}f_{n+a_{t}}=\sum\limits_{t\in T}f_{\left(
n+1\right)  +a_{t}}-\sum\limits_{t\in T}f_{n+a_{t}}%
\]
and%
\[
\sum\limits_{s\in S}f_{\left(  n-1\right)  +s}=\sum\limits_{s\in
S}\underbrace{f_{n+s-1}}_{=f_{n+s+1}-f_{n+s}}=\sum\limits_{s\in S}%
f_{n+s+1}-\sum\limits_{s\in S}f_{n+s}=\sum\limits_{s\in S}f_{\left(
n+1\right)  +s}-\sum\limits_{s\in S}f_{n+s}%
\]
), so it only remains to prove Assertion 1.

So let us prove Assertion 1. We have $\sum\limits_{t\in T}f_{n+a_{t}}%
=\sum\limits_{s\in S}f_{n+s}$, so that $\sum\limits_{t\in T}f_{n+a_{t}}%
-\sum\limits_{s\in S}f_{n+s}=0$. Thus,%
\begin{align*}
\sum\limits_{t\in T}f_{\left(  n+1\right)  +a_{t}}-\sum\limits_{s\in
S}f_{\left(  n+1\right)  +s}  &  =\sum\limits_{t\in T}f_{\left(  n+1\right)
+a_{t}}-\sum\limits_{s\in S}f_{\left(  n+1\right)  +s}-\phi\left(
\sum\limits_{t\in T}f_{n+a_{t}}-\sum\limits_{s\in S}f_{n+s}\right) \\
&  =\sum\limits_{t\in T}\left(  f_{n+a_{t}+1}-\phi f_{n+a_{t}}\right)
-\sum\limits_{s\in S}\left(  f_{n+s+1}-\phi f_{n+s}\right)  ,
\end{align*}
so that%
\begin{align}
&  \left\vert \sum\limits_{t\in T}f_{\left(  n+1\right)  +a_{t}}%
-\sum\limits_{s\in S}f_{\left(  n+1\right)  +s}\right\vert =\left\vert
\sum\limits_{t\in T}\left(  f_{n+a_{t}+1}-\phi f_{n+a_{t}}\right)
-\sum\limits_{s\in S}\left(  f_{n+s+1}-\phi f_{n+s}\right)  \right\vert
\nonumber\\
&  \leq\sum\limits_{t\in T}\left\vert f_{n+a_{t}+1}-\phi f_{n+a_{t}%
}\right\vert +\sum\limits_{s\in S}\left\vert f_{n+s+1}-\phi f_{n+s}\right\vert
\ \ \ \ \ \ \ \ \ \ \left(  \text{by the triangle inequality}\right)
\nonumber\\
&  =\sum\limits_{t\in T}\left(  \phi-1\right)  ^{n+a_{t}}+\sum\limits_{s\in
S}\left(  \phi-1\right)  ^{n+s}\ \ \ \ \ \ \ \ \ \ \left(  \text{by Theorem
2}\right) \nonumber\\
&  \leq\sum\limits_{t\in T}\left(  \phi-1\right)  ^{N+a_{t}}+\sum\limits_{s\in
S}\left(  \phi-1\right)  ^{N+s}\ \ \ \ \ \ \ \ \ \ \left(
\begin{array}
[c]{c}%
\text{since }\left(  \phi-1\right)  ^{n+a_{t}}\leq\left(  \phi-1\right)
^{N+a_{t}}\text{ and}\\
\left(  \phi-1\right)  ^{n+s}\leq\left(  \phi-1\right)  ^{N+s}\text{,
because}\\
n\geq N\text{ and }0\leq\phi-1\leq1
\end{array}
\right) \nonumber\\
&  =\sum\limits_{t\in T}\left(  \phi-1\right)  ^{N+a_{t}}+\sum\limits_{t\in
Z_{\nu}}\left(  \phi-1\right)  ^{t}\ \ \ \ \ \ \ \ \ \ \left(  \text{since
}S=\left\{  t-N\ \mid\ t\in Z_{\nu}\right\}  \right) \nonumber\\
&  =\sum\limits_{t\in T}\left(  \phi-1\right)  ^{N+a_{t}}+\sum\limits_{s\in
Z_{\nu}}\left(  \phi-1\right)  ^{s}=\left(  \phi-1\right)  ^{N}\sum
\limits_{t\in T}\left(  \phi-1\right)  ^{a_{t}}+\sum\limits_{s\in Z_{\nu}%
}\left(  \phi-1\right)  ^{s}\nonumber\\
&  \leq\left(  \phi-1\right)  ^{N}\sum\limits_{t\in T}\left(  \phi-1\right)
^{a_{t}}+\left(  \phi-1\right)  \label{Estimate}%
\end{align}
(since $\sum\limits_{s\in Z_{\nu}}\left(  \phi-1\right)  ^{s}\leq\phi-1$ by
Theorem 3, because $Z_{\nu}$ is a holey subset of $\mathbb{N}_{2}$).

Now, $\sum\limits_{t\in T}\left(  \phi-1\right)  ^{a_{t}}$ is a constant,
while $\left(  \phi-1\right)  ^{N}\rightarrow0$ for $N\rightarrow\infty$.
Hence, we can make the product $\left(  \phi-1\right)  ^{N}\sum\limits_{t\in
T}\left(  \phi-1\right)  ^{a_{t}}$ arbitrarily close to $0$ if we choose $N$
big enough. Since $\phi-1<1$, we have%
\begin{equation}
\left(  \phi-1\right)  ^{N}\sum\limits_{t\in T}\left(  \phi-1\right)  ^{a_{t}%
}+\left(  \phi-1\right)  <1 \label{BoundWIN}%
\end{equation}
if $\left(  \phi-1\right)  ^{N}\sum\limits_{t\in T}\left(  \phi-1\right)
^{a_{t}}$ is sufficiently close to $0$, what we can enforce by taking a high
enough $N$.

So let us take $N$ high enough so that (\ref{BoundWIN}) holds. Combined with
(\ref{Estimate}), it then yields%
\[
\left\vert \sum\limits_{t\in T}f_{\left(  n+1\right)  +a_{t}}-\sum
\limits_{s\in S}f_{\left(  n+1\right)  +s}\right\vert <1,
\]
which leads to $\left\vert \sum\limits_{t\in T}f_{\left(  n+1\right)  +a_{t}%
}-\sum\limits_{s\in S}f_{\left(  n+1\right)  +s}\right\vert =0$ (since
$\left\vert \sum\limits_{t\in T}f_{\left(  n+1\right)  +a_{t}}-\sum
\limits_{s\in S}f_{\left(  n+1\right)  +s}\right\vert $ is a nonnegative
integer). In other words, $\sum\limits_{t\in T}f_{\left(  n+1\right)  +a_{t}%
}=\sum\limits_{s\in S}f_{\left(  n+1\right)  +s}$. This completes the proof of
Assertion 1, and, with it, the solution of the problem.

\begin{center}
\textbf{References}
\end{center}

[1] Philip Matchett Wood, Doron Zeilberger, \textit{A translation method for
finding combinatorial bijections}, to appear in Annals of
Combinatorics.\newline\texttt{http://www.math.rutgers.edu/\symbol{126}%
zeilberg/mamarim/mamarimhtml/trans-method.html}


\end{document}